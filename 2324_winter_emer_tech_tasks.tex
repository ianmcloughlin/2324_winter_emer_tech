\documentclass{iansnotes}

\title{Tasks: Emerging Technologies}
\author{ian.mcloughlin@atu.ie}
\date{Last updated: \today}

\begin{document}
 
\maketitle
 
\begin{enumerate}
  \item The Collatz conjecture~\autocite{quantacollatz} is a famous unsolved problem in mathematics. The problem is to prove that if you start with any positive integer $x$ and repeatedly apply the function $f(x)$ below, you always get stuck in the repeating sequence $1,4,2,1,4,2,\ldots$ 
  
  $$ f(x) = \begin{cases}
    x \div 2 & \text{if } x \text{ is even} \\
    3x + 1              & \text{otherwise} 
  \end{cases}$$

  For example, starting with the value 10, which is an even number, we divide it by 2 to get 5.
  Then 5 is an odd number so, we multiply by 3 and add 1 to get 16.
  Then we repeatedly divide by 2 to get 8, 4, 2, 1.
  Once we are at 1, we go back to 4 and get stuck in the repeating sequence $4, 2, 1$ as we suspected.

  Your task is to verify, using Python, that the conjecture is true for the first 10,000 positive integers.
  
  \item Square roots are difficult to calculate. In Python, you typically use the power operator (a double asterisk) or a package such as \texttt{math}. In this task\autocite{golangnewton}, you should write a function \mintinline{python}{sqrt(x)} to approximate the square root of a floating point number $x$ without using the power operator or a package.
  
  Rather, you should use the Newton's method\autocite{newtonsqrt}. Start with an initial guess for the square root called $z_0$. You then repeatedly improve it using the following formula, until the difference between some previous guess $z_i$ and the next $z_{i+1}$ is less than some threshold, say $0.01$.
  
  $$ z_{i+1} = z_i -  \frac{z_i \times z_i - x}{2 z_i}$$
\end{enumerate}

\end{document}